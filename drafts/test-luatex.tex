\ifdefined\directlua
  \input luaotfload.sty
\fi
\font\tenrm="Alegreya" at 10pt
\tenrm

\input macros.tex

\parindent 0pt

Comme on dispose du moteur Luatex et des polices OTF, il est tout à fait possible d'écrire en français sans se préoccuper des accents. On peut écrire des expressions mathématiques très facilement. Par exemple, la fonction $f \colon D \to {\bf R}$, avec $D$ un sous-ensemble de $\bf R$, est dite continue si, pour tout $\epsilon > 0$ et pour tout $x \in D$, il existe un $\delta > 0$ tel que si $y \in D$ vérifie $|y - x| < \delta$, alors $$|f(y) - f(x)| < \epsilon.$$

Il est également possible d'inclure du code source dans un environnement de type {\tt verbatim} à l'aide d'une simple macro, comme illustré ci-dessous :
\bigskip

\verbatim
from sympy import isprime

def isperm(a, b):
    return sorted(str(a)) == sorted(str(b))

a, f = 1487, 1
d = 3330
while True:
    a += 3-f
    f = -f
    b, c = a+d, a+2*d
    if all(elt is True for elt in map(isprime, [a, b, c])) and \
       isperm(a, b) and isperm(b, c):
        break

print(str(a)+str(b)+str(c))
?endgroup

\bigskip
En somme, on peut très bien se passer de Latex et revenir vers les bases, \TeX !

Reste à voir : comment faire de beaux tableaux, inclure des figures, inclure les ligatures dans le code et comment gérer une bibliographie\dots
\bye
