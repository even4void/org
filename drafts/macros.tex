% From: Tim Morgan <morgan@uci-icsa>
% Produce verbatim listings of various sorts
 
\def\uncatcodespecials{\def\do##1{\catcode`##1=12 } \dospecials}
 
\def\setupverbatim{%
    \par \tt \spaceskip=0pt     % Make sure we get fixed tt spacing
    \obeylines\uncatcodespecials\obeyspaces\verbatimdefs
}
 
% This macro turns on verbatim mode until ?endverbatim is seen.
\def\verbatim{\begingroup \setupverbatim
  \parskip=0pt plus .05\baselineskip \parindent=0pt
   \catcode`\ =13 \catcode`\^^M=13 \catcode`\?=0
   \verbatimgobble}
{\catcode`\^^M=13{\catcode`\ =13\gdef\verbatimdefs{\def^^M{\ \par}\let =\ }}
  \gdef\verbatimgobble#1^^M{}}
 
% This defines ?endverbatim to end the group which begins with \verbatim
\let\endverbatim=\endgroup
 
% Input a file in verbatim mode.  Sometimes useful for including
% real-life examples into a paper.
\def\verbfile#1{\begingroup\setupverbatim
    \parskip=0pt plus .05\baselineskip \parindent=0pt
    \input#1 \endgroup
}
 
% This is the same as the above, but it adds line numbers to each
% line of the file printed.
\newcount\lineno
\def\listing#1{\lineno=0\begingroup\setupverbatim
    \parskip=0pt plus .05\baselineskip \parindent=20pt
    \everypar{\advance\lineno by 1 \llap{\the\lineno\ \ }}\input#1
    \endgroup
}
